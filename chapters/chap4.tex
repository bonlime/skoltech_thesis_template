
% тут сначала про данные и задачу, плюс много слов о том, почему это валидная задача
% потом таблички?? это в теории должна быть самая мясная часть работы и хорошо бы уже иметь хоть половину её написанной к пред защите, которая блин уже во вторник(((
% 
\chapter{Chapter 4. Experiments}

\subsection{Dataset} \label{subsec: imagenet}

% эту часть нужно перенести в главу 4 про эксперименты
% (сюда можно дописать еще общих слов из \cite{beyer2020_are_we_done}) для объема

Imagenet Large Scale Vision Recognition Challenge also known as ILSVRC2012 or simply Imagenet is ..... (тут чутка про историю датасета и разер и наврено можно написать что это подмножество). It has 1.3 millions training images divided into 1000 classes. Validation dataste consist of 50 thousand images, equally balanced between same 1000 classes. ILSVRC2012 is the most used dataset in computer vision (cite something, maybe after MNIST and CIFAR). In recent years with models becoming larger and demanding more and more data, it became a de-facto stardart benchmark for any new architecture. Since the breakthrought of AlexNet improving convolutional neural networks lead to improvements in other fields of computer vision. Typically researchers optimize models on classification problems using ImageNet (cite it) as proxy. It has been shown \cite{he2019bag_of_tricks} \cite{kornblith2019better} than improvments from Imagenet increase transfer learning performance other domains, for example object detection and semantic segmentation.


In this papepr we will use top1 1 classification accuracy on the ILSVRC2012 validation dataset as metric of model performance. It has a very strong correlation of $r=0.96$ \cite{kornblith2019better} between ImageNet accuracy and transfer accuracy, which makes it a well suitable metric.


(идея параграфа в том, что метрика не идеальная, но содйте. плюскуча рисерча накоплена в этой области, не выкидывать же его)
While some researches has objected that it's valid, it does correlate and this is what we are interested in (cite Are we done with Imagenet....) \cite{beyer2020_are_we_done}