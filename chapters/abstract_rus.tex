\selectlanguage{russian}

За последние несколько лет произошел огромный скачок в области разработки новых конволюционных архитектур для компьютерного зрения. Было представлено множество методов обучения в сочетании с модификациями самих моделей. В то время как проектирование архитектуры привлекает много внимания, оно часто скрывает улучшения, которые приходят от модифицированных методов регуляризации и дополнения. В нашей работе мы рассматриваем классическую архитектуру ResNet (He et al., 2015) и пытаемся отделить прирост производительности, который происходит от модификации модели, от влияния приемов обучения. Мы также изучаем аддитивную природу таких трюков, объединяя несколько существующих методов для достижения заметного улучшения. Удивительно, но наши результаты показывают, что стратегии обучения не менее важны, чем изменения в архитектуре. Обширные эксперименты по объединению этих стратегий улучшают точность ResNet50 с 76.5\% до 79.6\% без каких-либо изменений в модели. С помощью небольших изменений, которые почти не влияют на скорость прохода модели, качество может быть дополнительно увеличена до 81.2\%, что соответствует текущему уровню современных моделей. Главный вывод из нашей работы заключается в том, что модифицированная и пересмотренная ResNet-50 все еще может быть использована в качестве сильной базовой модели для будущих исследований и реальных приложений.
